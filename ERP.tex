\documentclass[paper=a4, fontsize=11pt, spanish]{scrartcl}
\usepackage{fourier}

\usepackage{sectsty}
\allsectionsfont{\normalfont\scshape}
\setlength\parindent{0pt}

\usepackage{fancyhdr}
\pagestyle{fancyplain}
\fancyhead{}
\fancyfoot[L]{}
\fancyfoot[C]{}
\fancyfoot[R]{\thepage}
\renewcommand{\headrulewidth}{0pt}
\renewcommand{\footrulewidth}{0pt}
\setlength{\headheight}{13.6pt}

\usepackage[spanish,es-noquoting,es-lcroman]{babel}
\usepackage[utf8]{inputenc}
\usepackage[T1]{fontenc}
\selectlanguage{spanish}

%----------------------------------------------------------------------------------------
%	TÍTULO
%----------------------------------------------------------------------------------------
% Título con las líneas horizontales, nombres y fecha.

\newcommand{\horrule}[1]{\rule{\linewidth}{#1}}

\title{
	\normalfont \normalsize
	\textsc{Universidad de Granada.} \\ [25pt]
	\horrule{0.5pt} \\[0.4cm]
	\huge ERP \\
	\horrule{2pt} \\[0.5cm]
}

\author{Marina Estévez Almenzar\\
	\and
	Jose Pimentel Mesones\\
	\and
	Sergio Padilla López\\
	\and
	Iván Calle Gil\\}

\date{\normalsize\today}


%----------------------------------------------------------------------------------------
%	DOCUMENTO
%----------------------------------------------------------------------------------------

\begin{document}
	\maketitle
	
	\section{Funcionalidad del Sistema}
	
	\subsection{Funcionalidad “Gestión de tareas”}
	
	\subsubsection{Descripción}
	\setlength{\parindent}{3em} La gestión de tareas permite una supervisión cómoda y eficiente de una lista de trabajos asignables a los empleados para la estructuración interna de la empresa y el control del trabajo personal y colectivo.
	
	Para llevar a cabo esta gestión, el administrador crea las tareas y se las asigna a sus empleados (o a él mismo). Además permite asignar una fecha de vencimiento de la tarea, tiempo empleado en llevarla a cabo, registro de horas trabajadas en la misma y comunicación entre administrador y empleado mediante comentarios en la propia tarea.
	
	\subsection{Funcionalidad “Gestión de facturas y productos”}
	
	\subsubsection{Descripción}
	\setlength{\parindent}{3em} La gestión de facturas y productos permite a la empresa un seguimiento fiable y un control eficiente de los productos ofrecidos por la misma, a la vez que proporciona una buena ordenación de las facturas emitidas en la compra de dichos productos. Como consecuencia, esta gestión también proporcionará las relaciones que se establecen entre los productos y las facturas.
	Para llevar a cabo esta gestión, el administrador creará los productos que desea ofrecer al cliente, añadiendo la opción de poder consultarlos, modificarlos e incluso eliminarlos de la lista de productos ofrecidos. En cuanto a las facturas, también el administrador será el responsable de crearlas y consultarlas, Cabe la posibilidad de añadir la opción de impresión de facturas.
	
	\subsubsection{Requisitos funcionales}
	\setlength{\parindent}{0em}
	\textbf{RF1: \textit{Crear tarea}}
	\setlength{\parindent}{2em}
	
	Actor: Administrador
	
	Entrada: RD1
	
	Procesamiento: Crea una nueva tarea
	
	Salida: ID tarea
	
	\setlength{\parindent}{0em}
	\textbf{RF2: \textit{Añadir subtarea}}
	\setlength{\parindent}{2em}
	
	Actor: Administrador
	
	Entrada: RD2
	
	Procesamiento: Añadir tarea a la lista de subtareas
	
	Salida; Ninguna
	
	\setlength{\parindent}{0em}
	\textbf{RF3: \textit{Asignar fecha a tarea}}
	\setlength{\parindent}{2em}
	
	Actor: Administrador
	
	Entrada: RD3
	
	Procesamiento: Se asigna fecha de vencimiento a la tarea en cuestión
	
	Salida: Ninguna
	
	\setlength{\parindent}{0em}
	\textbf{RF4: \textit{Aumentar horas de trabajo en la tarea}}
	\setlength{\parindent}{2em}
	
	Actor: Empleado
	
	Entrada: RD4
	
	Procesamiento: Se suman las horas a las ya trabajadas y se restan de las estimadas para completar la tarea
	
	Salida: Ninguna
	
	\setlength{\parindent}{0em}
	\textbf{RF5: \textit{Añadir comentario}}
	\setlength{\parindent}{2em}
	
	Actor: Administrador o empleado
	
	Entrada: RD6
	
	Procesamiento: Añade un comentario a la tarea
	
	Salida: Ninguna
	
	\setlength{\parindent}{0em}
	\textbf{RF6: \textit{Borrar comentario}}
	\setlength{\parindent}{2em}
	
	Actor: Administrador o empleado
	
	Entrada: RD7
	
	Procesamiento: Borra el comentario especificado
	
	Salida: Ninguna
	
	\setlength{\parindent}{0em}
	\textbf{RF7: \textit{Asignar tarea}}
	\setlength{\parindent}{2em}
	
	Actor: Administrador o empleado
	
	Entrada: RD5
	
	Procesamiento: Asignación de tarea a la persona indicada
	
	Salida: Ninguna
	
	\setlength{\parindent}{0em}
	\textbf{RF8: \textit{Borrar tarea}}
	\setlength{\parindent}{2em}
	
	Actor: Administrador o empleado
	
	Entrada: RD2
	
	Procesamiento: Borrar tarea y subtareas indicadas
	
	Salida: Ninguna, o mensaje de error.
	
	\setlength{\parindent}{0em}
	\textbf{RF9: \textit{Modificar tarea}}
	\setlength{\parindent}{2em}
	
	Actor: Administrador o empleado
	
	Entrada: RD8
	
	Procesamiento: Modifica la tarea indicada
	
	Salida: Ninguna, o mensaje de error.
	
	\setlength{\parindent}{0em}
	\textbf{RF15: \textit{Crear producto}}
	\setlength{\parindent}{2em}
	
	Actor: Administrador
	
	Entrada: RD17
	
	Procesamiento: Crea un nuevo producto
	
	Salida: Ninguna (si falta algún dato de entrada, devuelve un mensaje de error)
	
	\setlength{\parindent}{0em}
	\textbf{RF16: \textit{Consultar producto}}
	\setlength{\parindent}{2em}
	
	Actor: Administrador o empleado
	
	Entrada: RD18
	
	Procesamiento: Consultar el estado de un producto
	
	Salida: ID, nombre, descripción, importe del producto, número de ventas (si el producto no existe, devuelve un mensaje de error)
	
	\setlength{\parindent}{0em}
	\textbf{RF17: \textit{Modificar producto}}
	\setlength{\parindent}{2em}
	
	Actor: Administrador
	
	Entrada: RD19
	
	Procesamiento: Se modifican los datos del producto en cuestión
	
	Salida: Ninguna (si el producto no existe, devuelve un mensaje de error)
	
	\setlength{\parindent}{0em}
	\textbf{RF18: \textit{Ordenar productos}}
	\setlength{\parindent}{2em}
	
	Actor: Administrador o empleado
	
	Entrada: Ninguna
	
	Procesamiento: Ordena los productos del más vendido al menos vendido
	
	Salida: Lista de productos (ID y nombre del producto)
	
	\setlength{\parindent}{0em}
	\textbf{RF19: \textit{Eliminar producto}}
	\setlength{\parindent}{2em}
	
	Actor: Administrador
	
	Entrada: RD20
	
	Procesamiento: Se elimina el producto en cuestión del conjunto de productos 
	
	Salida: Ninguna (si el producto no existe, devuelve un mensaje de error)
	
	\setlength{\parindent}{0em}
	\textbf{RF20: \textit{Crear factura}}
	\setlength{\parindent}{2em}
	
	Actor: Administrador o empleado
	
	Entrada: RD21
	
	Procesamiento: Crea una factura de la compra de algún cliente 
	
	Salida: Ninguna (si falta algún dato de entrada, devuelve un mensaje de error)
	
	\setlength{\parindent}{0em}
	\textbf{RF21: \textit{Consultar factura}}
	\setlength{\parindent}{2em}
	
	Actor: Administrador o empleado
	
	Entrada: RD22
	
	Procesamiento: Consulta una de las facturas ya creadas 
	
	Salida: ID de factura, cliente correspondiente, fecha de emisión e importe total
	
	\setlength{\parindent}{0em}
	\textbf{RF22: \textit{Imprimir factura (opcional)}}
	\setlength{\parindent}{2em}
	
	Actor: Administrador o empleado
	
	Entrada: RD23
	
	Procesamiento: Se imprime la factura en cuestión
	
	Salida: Ninguna (si la factura no existe, devuelve un mensaje de error)
	
	\section{Requisitos de datos}
	\setlength{\parindent}{0em}
	\textbf{RD1: \textit{Datos opcionales para crear una tarea}}
	\setlength{\parindent}{2em}
	Título [cadena de caracteres]
	
	Descripción [Cadena de caracteres]
	
	Tiempo estimado [número real]
	
	\setlength{\parindent}{0em}
	\textbf{RD2: \textit{Datos necesarios para identificar una tarea}}
	\setlength{\parindent}{2em}
	\begin{itemize}
		\item ID tarea
	\end{itemize}
	
	\setlength{\parindent}{0em}
	\textbf{RD3: \textit{Datos necesarios para asignar fecha a una tarea}}
	\setlength{\parindent}{2em}
	\begin{itemize}
		\item Fecha
	\end{itemize}
	
	\setlength{\parindent}{0em}
	\textbf{RD4: \textit{Datos necesarios para aumentar las horas de trabajo en una tarea}}
	\setlength{\parindent}{2em}
	\begin{itemize}
		\item Horas trabajadas
	\end{itemize}
	
	\setlength{\parindent}{0em}
	\textbf{RD5: \textit{Datos necesarios para asignar tarea}}
	\setlength{\parindent}{2em}
	\begin{itemize}
		\item ID empleado
	\end{itemize}
	
	\setlength{\parindent}{0em}
	\textbf{RD6: \textit{Datos necesarios para añadir comentario}}
	\setlength{\parindent}{2em}
	\begin{itemize}
		\item Comentario [cadena de texto]
	\end{itemize}
	
	\setlength{\parindent}{0em}
	\textbf{RD7: \textit{Datos necesarios para borrar comentario}}
	\setlength{\parindent}{2em}
	\begin{itemize}
		\item ID comentario
	\end{itemize}
	
	\setlength{\parindent}{0em}
	\textbf{RD8: \textit{Datos almacenados en TAREA}}
	\setlength{\parindent}{2em}
	\begin{itemize}
		\item ID tarea
		\item Fecha de vencimiento
		\item ID tarea\_padre
		\item Horas estimadas 
		\item Horas trabajas 
		\item ID registro
	\end{itemize}
	
	\setlength{\parindent}{0em}
	\textbf{RD9: \textit{Datos almacenados en COMENTARIOS}}
	\setlength{\parindent}{2em}
	\begin{itemize}
		\item ID comentario
		\item texto [cadena de caracteres]
		\item ID Tarea
	\end{itemize}
	
	\setlength{\parindent}{0em}
	\textbf{RD10: \textit{Datos almacenados en REGISTRO}}
	\setlength{\parindent}{2em}
	\begin{itemize}
		\item ID registro
		\item ID empleado
		\item Horas trabajadas
		\item Descripción [cadena de texto]
		\item Fecha
	\end{itemize}
	
	\setlength{\parindent}{0em}
	\textbf{RD17: \textit{Datos necesarios para crear un producto}}
	\setlength{\parindent}{2em}
	\begin{itemize}
		\item Nombre del producto [cadena de caracteres]
		\item Descripción [cadena de caracteres]
		\item Importe
	\end{itemize}
	
	\setlength{\parindent}{0em}
	\textbf{RD18: \textit{Datos necesarios para consultar un producto}}
	\setlength{\parindent}{2em}
	\begin{itemize}
		\item ID del producto
	\end{itemize}
	
	\setlength{\parindent}{0em}
	\textbf{RD19: \textit{Datos necesarios para modificar un producto}}
	\setlength{\parindent}{2em}
	\begin{itemize}
		\item ID del producto
		\item Dato a modificar (nombre del producto, descripción, importe)
	\end{itemize}
	
	\setlength{\parindent}{0em}
	\textbf{RD20: \textit{Datos necesarios para eliminar un producto}}
	\setlength{\parindent}{2em}
	\begin{itemize}
		\item ID de producto
	\end{itemize}
	
	\setlength{\parindent}{0em}
	\textbf{RD21: \textit{Datos necesarios para crear una factura}}
	\setlength{\parindent}{2em}
	\begin{itemize}
		\item ID de factura
		\item Fecha de emisión [DD-MM-AAAA]
		\item ID del cliente involucrado
		\item Importe total [número real]
	\end{itemize}
	
	\setlength{\parindent}{0em}
	\textbf{RD22: \textit{Datos necesarios para consultar una factura}}
	\setlength{\parindent}{2em}
	\begin{itemize}
		\item ID de factura
	\end{itemize}
	
	\setlength{\parindent}{0em}
	\textbf{RD23: \textit{Datos necesarios para imprimir una factura}}
	\setlength{\parindent}{2em}
	\begin{itemize}
		\item ID de factura
	\end{itemize}
	
	\setlength{\parindent}{0em}
	\textbf{RD24: \textit{Datos almacenados en PRODUCTOS}}
	\setlength{\parindent}{2em}
	\begin{itemize}
		\item ID de producto
		\item Nombre del producto [cadena de caracteres]
		\item Descripción [cadena de caracteres]
		\item Importe [número real]
		\item Número de ventas [número entero]
	\end{itemize}
	
	\setlength{\parindent}{0em}
	\textbf{RD25: \textit{Datos almacenados en FACTURAS}}
	\setlength{\parindent}{2em}
	\begin{itemize}
		\item ID de factura
		\item Fecha de emisión [DD-MM-AAAA]
		\item ID cliente involucrado
		\item Importe total [número real]
	\end{itemize}
	
	\setlength{\parindent}{0em}
	\textbf{RD26: \textit{Datos almacenados en COMPRAS}}
	\setlength{\parindent}{2em}
	\begin{itemize}
		\item ID de factura
		\item ID de producto
	\end{itemize}
	
	\section{Restricciones Semánticas}
	\setlength{\parindent}{0em}
	\textbf{RS1: \textit{Tareas y subtareas}}
	\setlength{\parindent}{2em}
	
	Las subtareas son tareas en sí, tienen su propia id, diferente de la tarea de la que dependen la cual se refleja en la base de datos para saber que es subtarea.
	
	\setlength{\parindent}{0em}
	\textbf{RS2: \textit{Asignación de tareas}}
	\setlength{\parindent}{2em}
	
	Cualquier empleado puede asignar tareas.
	
	\setlength{\parindent}{0em}
	\textbf{RS3: \textit{Administrador y empleados}}
	\setlength{\parindent}{2em}
	
	El administrador es un empleado más pero con algunos privilegios especiales.
	
	\setlength{\parindent}{0em}
	\textbf{RS4: \textit{Creación de tareas}}
	\setlength{\parindent}{2em}
	
	El administrador es el único que puede crear tareas.

	\setlength{\parindent}{0em}
	\textbf{RS10: \textit{Creación y eliminación de productos}}
	\setlength{\parindent}{2em}

	El administrador es el único que puede crear y eliminar productos. 

	\setlength{\parindent}{0em}
	\textbf{RS11: \textit{Consultar, modificar y ordenar productos}}
	\setlength{\parindent}{2em}
	Tanto el administrador como los empleados pueden consultar, modificar y ordenar productos.
	
	\setlength{\parindent}{0em}
	\textbf{RS12: \textit{Creación de facturas}}
	\setlength{\parindent}{2em}
	
	El administrador es el único capaz de crear facturas.
	
	
	\setlength{\parindent}{0em}
	\textbf{RS13: \textit{Consultar e imprimir(*) facturas}}
	\setlength{\parindent}{2em}
	
	Tanto el administrador como los empleados pueden consultar e imprimir facturas.
	
	\setlength{\parindent}{0em}
	\textbf{RS14: \textit{Facturas y productos eliminados}}
	\setlength{\parindent}{2em}
	
	Al eliminar un producto, no desaparecen las posibles relaciones que tenga dicho producto con las facturas de compras anteriores del mismo: el producto eliminado no desaparece de la tabla que relaciona cada factura con los productos involucrados.

\end{document}
