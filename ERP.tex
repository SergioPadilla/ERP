\documentclass[paper=a4, fontsize=11pt, spanish]{scrartcl}
\usepackage{fourier}

\usepackage{sectsty}
\allsectionsfont{\normalfont\scshape}
\setlength\parindent{0pt}

\usepackage{fancyhdr}
\pagestyle{fancyplain}
\fancyhead{}
\fancyfoot[L]{}
\fancyfoot[C]{}
\fancyfoot[R]{\thepage}
\renewcommand{\headrulewidth}{0pt}
\renewcommand{\footrulewidth}{0pt}
\setlength{\headheight}{13.6pt}

\usepackage[spanish,es-noquoting,es-lcroman]{babel}
\usepackage[utf8]{inputenc}
\usepackage[T1]{fontenc}
\selectlanguage{spanish}

%----------------------------------------------------------------------------------------
%	TÍTULO
%----------------------------------------------------------------------------------------
% Título con las líneas horizontales, nombres y fecha.

\newcommand{\horrule}[1]{\rule{\linewidth}{#1}}

\title{
  \normalfont \normalsize
  \textsc{Universidad de Granada.} \\ [25pt]
  \horrule{0.5pt} \\[0.4cm]
  \huge ERP \\
  \horrule{2pt} \\[0.5cm]
}

\author{Marina Estévez Almenzar\\
\and
Jose Pimentel Mesones\\
\and
Sergio Padilla López\\
\and
Iván Calle Gil\\}

\date{\normalsize\today}


%----------------------------------------------------------------------------------------
%	DOCUMENTO
%----------------------------------------------------------------------------------------

\begin{document}
\maketitle

\section{Funcionalidad del Sistema}

\subsection{Funcionalidad “Gestión de Clientes”}

\subsubsection{Descripción}
\setlength{\parindent}{3em} La gestión de clientes permite manejar de forma cómoda y eficiente los datos de los clientes que han contratado algún servicio con la empresa.

Para llevar a cabo esta gestión, el administrador será el único que pueda manejar la información de los clientes, pudiendo dar de alta , modificarla o borrarla.
Todos los empleados podrán consultar la información de los clientes

\subsubsection{Requisitos funcionales}
\setlength{\parindent}{0em}
\textbf{RF10: \textit{Crear cliente}}
\setlength{\parindent}{2em}

Actor: Administrador

Entrada: RD1

Procesamiento: Crea un nuevo cliente

Salida: RD2

\setlength{\parindent}{0em}
\textbf{RF12: \textit{Borrar cliente}}
\setlength{\parindent}{2em}

Actor: Administrador

Entrada: RD3

Procesamiento:Se elimina los datos relacionados con el cliente de ID especificada

Salida; Ninguna (si no existe el ID, devuelve mensaje de error)

\setlength{\parindent}{0em}
\textbf{RF13: \textit{Modificar cliente}}
\setlength{\parindent}{2em}

Actor: Administrador

Entrada: RD3, RD4

Procesamiento: Se modifica los datos suministrados mediante que asociados al cliente con ID especificado en RD3

Salida: Ninguna (si no existe el ID, devuelve mensaje de error)

\setlength{\parindent}{0em}
\textbf{RF14: \textit{Consultar cliente}}
\setlength{\parindent}{2em}

Actor: Administrador, Empleado

Entrada: RD3

Procesamiento: Se muestra la información almacenado del cliente con identificador proporcionado por RD3

Salida:  RD2 (si no existe el ID, devuelve mensaje de error)

\setlength{\parindent}{0em}
\textbf{RF15: \textit{Clasificar clientes}}
\setlength{\parindent}{2em}

Actor: Administrador, Empleado

Entrada: RD5

Procesamiento:  Se busca entre todos los clientes almacenados aquellos que cumplan las condiciones impuestas por RD5

Salida: RD6 (pudiendo aparecer una lista vacia)


\end{document}
