\documentclass[paper=a4, fontsize=11pt, spanish]{scrartcl}
\usepackage{fourier}

\usepackage{sectsty}
\allsectionsfont{\normalfont\scshape}
\setlength\parindent{0pt}

\usepackage{fancyhdr}
\pagestyle{fancyplain}
\fancyhead{}
\fancyfoot[L]{}
\fancyfoot[C]{}
\fancyfoot[R]{\thepage}
\renewcommand{\headrulewidth}{0pt}
\renewcommand{\footrulewidth}{0pt}
\setlength{\headheight}{13.6pt}

\usepackage[spanish,es-noquoting,es-lcroman]{babel}
\usepackage[utf8]{inputenc}
\usepackage[T1]{fontenc}
\selectlanguage{spanish}

%----------------------------------------------------------------------------------------
%	TÍTULO
%----------------------------------------------------------------------------------------
% Título con las líneas horizontales, nombres y fecha.

\newcommand{\horrule}[1]{\rule{\linewidth}{#1}}

\title{
  \normalfont \normalsize
  \textsc{Universidad de Granada.} \\ [25pt]
  \horrule{0.5pt} \\[0.4cm]
  \huge ERP \\
  \horrule{2pt} \\[0.5cm]
}

\author{Marina Estévez Almenzar\\
\and
Jose Pimentel Mesones\\
\and
Sergio Padilla López\\
\and
Iván Calle Gil\\}

\date{\normalsize\today}


%----------------------------------------------------------------------------------------
%	DOCUMENTO
%----------------------------------------------------------------------------------------

\begin{document}
\maketitle

\section{Funcionalidad del Sistema}

\subsection{Funcionalidad “Gestión de tareas”}

\subsubsection{Descripción}
\setlength{\parindent}{3em} La gestión de tareas permite una supervisión cómoda y eficiente de una lista de trabajos asignables a los empleados para la estructuración interna de la empresa y el control del trabajo personal y colectivo.

	Para llevar a cabo esta gestión, el administrador crea las tareas y se las asigna a sus empleados (o a él mismo). Además permite asignar una fecha de vencimiento de la tarea, tiempo empleado en llevarla a cabo, registro de horas trabajadas en la misma y comunicación entre administrador y empleado mediante comentarios en la propia tarea.

\subsubsection{Requisitos funcionales}
\setlength{\parindent}{0em}
\textbf{RF1: \textit{Crear tarea}}
\setlength{\parindent}{2em}

Actor: Administrador

Entrada: RD1

Procesamiento: Crea una nueva tarea

Salida: ID tarea

\setlength{\parindent}{0em}
\textbf{RF2: \textit{Añadir subtarea}}
\setlength{\parindent}{2em}

Actor: Administrador

Entrada: RD2

Procesamiento: Añadir tarea a la lista de subtareas

Salida; Ninguna

\setlength{\parindent}{0em}
\textbf{RF3: \textit{Asignar fecha a tarea}}
\setlength{\parindent}{2em}

Actor: Administrador

Entrada: RD3

Procesamiento: Se asigna fecha de vencimiento a la tarea en cuestión

Salida: Ninguna

\setlength{\parindent}{0em}
\textbf{RF4: \textit{Aumentar horas de trabajo en la tarea}}
\setlength{\parindent}{2em}

Actor: Empleado

Entrada: RD4

Procesamiento: Se suman las horas a las ya trabajadas y se restan de las estimadas para completar la tarea

Salida: Ninguna

\setlength{\parindent}{0em}
\textbf{RF5: \textit{Añadir comentario}}
\setlength{\parindent}{2em}

Actor: Administrador o empleado

Entrada: RD6

Procesamiento: Añade un comentario a la tarea

Salida: Ninguna

\setlength{\parindent}{0em}
\textbf{RF6: \textit{Borrar comentario}}
\setlength{\parindent}{2em}

Actor: Administrador o empleado

Entrada: RD7

Procesamiento: Borra el comentario especificado

Salida: Ninguna

\setlength{\parindent}{0em}
\textbf{RF7: \textit{Asignar tarea}}
\setlength{\parindent}{2em}

Actor: Administrador o empleado

Entrada: RD5

Procesamiento: Asignación de tarea a la persona indicada

Salida: Ninguna

\section{Requisitos de datos}
\setlength{\parindent}{0em}
\textbf{RD1: \textit{Datos opcionales para crear una tarea}}
\setlength{\parindent}{2em}
Título [cadena de caracteres]

Descripción [Cadena de caracteres]

Tiempo estimado [número real]

\setlength{\parindent}{0em}
\textbf{RD2: \textit{Datos necesarios para asignar subtarea}}
\setlength{\parindent}{2em}
\begin{itemize}
\item ID tarea
\end{itemize}

\setlength{\parindent}{0em}
\textbf{RD3: \textit{Datos necesarios para asignar fecha a una tarea}}
\setlength{\parindent}{2em}
\begin{itemize}
\item Fecha [DD-MM-AAAA]
\end{itemize}

\setlength{\parindent}{0em}
\textbf{RD4: \textit{Datos necesarios para aumentar las horas de trabajo en una tarea}}
\setlength{\parindent}{2em}
\begin{itemize}
\item Horas trabajadas
\end{itemize}

\setlength{\parindent}{0em}
\textbf{RD5: \textit{Datos necesarios para asignar tarea}}
\setlength{\parindent}{2em}
\begin{itemize}
\item ID trabajador (puede ser el administrador)
\end{itemize}

\setlength{\parindent}{0em}
\textbf{RD6: \textit{Datos necesarios para añadir comentario}}
\setlength{\parindent}{2em}
\begin{itemize}
\item Comentario [cadena de texto]
\end{itemize}

\setlength{\parindent}{0em}
\textbf{RD7: \textit{Datos necesarios para borrar comentario}}
\setlength{\parindent}{2em}
\begin{itemize}
\item ID comentario
\end{itemize}

\setlength{\parindent}{0em}
\textbf{RD8: \textit{Datos almacenados en TAREA}}
\setlength{\parindent}{2em}
\begin{itemize}
\item ID tarea
\item Fecha de vencimiento [DD-MM-AAAA]
\item Subtareas [Lista de IDs de tareas]
\item Comentarios [Lista de IDs de comentarios]
\item Horas estimadas [HH-MM]
\item Horas trabajas [HH-MM]
\item ID registro [RD10]
\end{itemize}

\setlength{\parindent}{0em}
\textbf{RD9: \textit{Datos almacenados en COMENTARIOS}}
\setlength{\parindent}{2em}
\begin{itemize}
\item ID comentario
\item texto [cadena de caracteres]
\end{itemize}

\setlength{\parindent}{0em}
\textbf{RD10: \textit{Datos almacenados en REGISTRO}}
\setlength{\parindent}{2em}
\begin{itemize}
\item ID registro
\item ID empleado
\item Horas trabajadas [HH-MM]
\item Descripción [cadena de texto]
\item Fecha [DD-MM-AAAA]
\end{itemize}

\section{Restricciones Semánticas}
\setlength{\parindent}{0em}
\textbf{RS1: \textit{Tareas y subtareas}}
\setlength{\parindent}{2em}
Cada tarea puede contener a su vez subtareas (lista de ids con las tareas que lo forman). Las subtareas son tareas en sí, tienen su propia id, diferente de la tarea de la que dependen, además del resto de datos de una tarea.

\setlength{\parindent}{0em}
\textbf{RS2: \textit{Asignación de tareas}}
\setlength{\parindent}{2em}
Cualquier empleado puede asignar una tareas.

\setlength{\parindent}{0em}
\textbf{RS3: \textit{Administrador y empleados}}
\setlength{\parindent}{2em}
El administrador es un empleado más pero con algunos privilegios especiales.

\setlength{\parindent}{0em}
\textbf{RS4: \textit{Creación de tareas}}
\setlength{\parindent}{2em}
El administrador es el único que puede crear tareas.


\end{document}
